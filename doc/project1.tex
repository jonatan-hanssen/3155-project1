\documentclass{article}
	\usepackage[utf8]{inputenc}
	\usepackage{mathtools}
	\usepackage{amssymb}
	\usepackage{amsmath}
	\usepackage{bm}
	\usepackage{float}
	\usepackage{listings}
	\usepackage{nicefrac}
	\usepackage{xcolor}
	\usepackage[T1]{fontenc}
	\usepackage{newtxmath,newtxtext}
	\author{Jonatan Hoffmann Hanssen \and Eric Emanuel Reber}
	\title{Project 1 FYS-STK3155}

\begin{document}
	\maketitle

\section*{Abstract}

\section*{a}

We have defined $\bm{y}$ as a function of a matrix multiplication $X\bm{\beta}$
plus an error vector $\bm{\epsilon}$. This means that each element of the vector
$\bm{y}$ can be expressed as follows:

\[ y_i = \sum_j x_{ij}\beta_j + \epsilon_i \]

If we take the expectation value of this expression we get the following:

\[ E[y_i] = E\left[\sum_j x_{ij}\beta_j + \epsilon_i\right] \]
\[ E[y_i] = E\left[\sum_j x_{ij}\beta_j\right] + E[\epsilon_i] \]

However, the elements of $X$ are not stochastic, and neither are the elements of $\beta$
the first expectation value is simply the sum itself. Furthermore, $\epsilon$ is 
explicitly defined as a normal distribution $N(0,\sigma^2)$, and will by definition have
the expectation value $0$. Therefore, we end up with the final expression:

\[ E[y_i] = \sum_j x_{ij}\beta_j = \bm{X}_{i,*}\bm{\beta}\]

We can use expectation values to calculate the variance as well:

\[ var[y_i] = E\left[(y_i - E[y_i])^2\right] \]

\end{document}
